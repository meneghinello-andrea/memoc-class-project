%-------------------------------------------------
%
% Problem.tex
%
%------------------------------------------------
\section{Prima parte}
\label{pt1:problem}
Il problema può essere assimilato al problema del commesso viaggiatore (TSP) il quale deve visitare un insieme di città attraversandole \keyword{una ed una sola} volta. Pensando alle città (fori nel caso concreto) come ad un grafo si chiede di determinare il miglior cammino hamiltoniano che lo attraversa.

La problematica può essere espressa attraverso il seguente modello di programmazione lineare intera:

\begin{align}
MIN: &\sum_{i,j:(i,j) \in A} C_{i,j}x_{i,j}                       &  \nonumber           & \nonumber \\
S.T. &\sum_{j:(0,j) \in A} x_{0,j}                                & = \left | N \right | & \nonumber \\
     &\sum_{i:(i,k) \in A} x_{i,k} - \sum_{j_(k,j) \in A} x_{k,j} & = 1                  & \forall k \in N \textbackslash \left \{ 0 \right \} \\     
     &\sum_{j:(i,j) \in A} y_{i,j}                                & = 1                  & \forall i \in N \\     
     &\sum_{i:(i,j) \in A} y_{i,j}                                & = 1                  & \forall j \in N \\
     &\sum_{i:(i,j) \in A} y_{i,j}                                & \le \left | N \right |y_{i,j} \\
     &x_{i,j} \in \mathbb{Z}_{+}                                  & \nonumber            & \nonumber \\
     &y_{i,j} \in \left \{ 0, 1 \right \}                         & \nonumber            & \nonumber \\
\end{align}