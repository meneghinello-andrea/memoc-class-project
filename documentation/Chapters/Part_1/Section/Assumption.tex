%-------------------------------------------------
%
% Assumption.tex
%
%------------------------------------------------
\section[Assunzioni]{assunzioni}
\label{pt1:assumption}
Posso assumere senza alcuna perdita di generalità nel problema che il grafo G sia completo. Dati quindi due nodi $i$ e $j$ saranno contemporaneamente presenti l'arco $i \rightarrow j$ e l'arco $j \rightarrow j$.

Al fine di ottenere migliori prestazioni si è deciso di omettere gli archi che collegano un nodo con se stesso in quanto non sono portatori di informazione utile al problema.

In conclusione la cardinalità finale della matrice associata al grafo risulta essere:

\begin{equation}
\left|A\right| = \frac{\left|N\right|\times\left(\left|N\right|-1\right)}{2} - \left|N\right|
\end{equation}

Si è infine assunto che i costi $c_{i,j}$ corrispondano alla distanza euclidea tra il nodo $i$ ed il nodo $j$ e che $c_{i,j}=c_{j,i} \forall (i,j)\in A$ ovvero il \keyword{TSP sia simmetrico}.

Viene ora fornito un esempio di matrice associata al grafo del problema.

\begin{equation}
A=
\begin{bmatrix} 
0                        & c_{0,1}                  & c_{0,2}                  & \cdots & c_{0,\left|N\right|}                  \\
c_{1,0}                  & 0                        & c_{1,2}                  & \cdots & c_{1,\left|N\right|}                  \\
\vdots                   & \vdots                   & \vdots                   & \vdots & \vdots                                \\
c_{\left|N\right| - 1,0} & c_{\left|N\right| - 1,1} & c_{\left|N\right| - 1,2} & 0      & c_{\left|N\right| - 1,\left|N\right|} \\
c_{\left|N\right|,0}     & c_{\left|N\right|,1}     & c_{\left|N\right|,2}     & \cdots & 0                                     \\
\end{bmatrix}
\end{equation}