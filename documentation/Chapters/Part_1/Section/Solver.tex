%-------------------------------------------------
%
% Solver.tex
%
%------------------------------------------------
\section[Il risolutore]{il risolutore}
\label{pt1:solver}
Il programma implementato per determinare la soluzione del problema illustrato in Sezione \ref{pt1:problem} è stato implementato utilizzando il linguaggio $C++$ con chiamate alle librerie IBM CPlex \english{Callable Library}.

Viene ora fornita la struttura dei \english{files} che compongono tale programma.

\dirtree{%
.1 cplex\_solver.
.2 src.
.3 header.
.4 macro.h.
.5 cpxmacro.h.
.4 solver.
.5 TSPSolver.h.
.3 source.
.4 solver.
.5 TSPSolver.cpp.
.2 main.cpp.
.2 makefile.
}

La componente più importante del programma è la classe \texttt{TSPSolver} essa ha il compito di leggere i dati e configurare l'ambiente CPlex prima di lanciare l'algoritmo di risoluzione.

I metodi che la compongono degni di interesse sono i seguenti:

\dirtree{%
.1 TSPSolver.
.2 protected.
.3 findIndex.
.3 readProblemDatas.
.3 setObjectiveFunction.
.3 setConstraint\_1.
.3 setConstraint\_2.
.3 setConstraint\_3.
.3 setConstraint\_4.
.3 setConstraint\_5.
.2 public.
.3 resolve.
}

\subsection[Caricamento dati]{caricamento dati}
\label{pt1:solver:datas}
Data la possibilità di gestire problemi di grosse dimensioni ed al fine di ottimizzare la risorsa memoria si è deciso di generare, e conseguentemente caricare nel programma \english{solver} solamente metà matrice associata al grafo; la matrice triangolare superiore. E' compito del programma ripristinare i valori assenti.

I dati del problema sono memorizzati, dal metodo \texttt{readProblemDatas}, all'interno di un vettore e sono resi accessibili in forma matriciale dal metodo \texttt{findIndex}. Tale metodo virtualizza la matrice associata al grafo permettendo alle rimanenti parti, che compongono il solver, di poter operare su tale matrice nascondendo la sua linearizzazione.

Tale virtualizzazione è resa possibile dal seguente algoritmo per il calcolo dell'indice del vettore linearizzato.

\begin{lstlisting}[frame=single]
int TSPSolver::findIndex(int row, int col)
{
	//Numero di valori assenti prima della posizione (row, col)
	int empty_space = 0;
	
	//Indice a cui si trova il valore puntato da (row, col)
	int index = 0;
	
	//Indice di inizio riga della matrice virtuale
	int row_start = 0;
	
	if(row < column)
	{
		//Calcoli per la matrice triangolare inferiore
		row_start = (N * row) - ((row * (row + 1)) / 2;
		empty_space = (row + 1);
		index = row_start - empty_space + col;
	}
	else
	{
		//Calcoli per la matrice triangolare superiore
		row_start = (N * col) - ((col * (col + 1)) / 2;
		empty_space = (col + 1);
		index = row_start - empty_space + row;
	}
	
	return index;
}
\end{lstlisting}

\subsection[Settaggio variabili]{settaggio variabili}
\label{pt1:solver:variables}
Al fine di poter aggiungere variabili al modello in CPlex, occorre richiamare il metodo \texttt{CPXnewcols}. Esso permette il settaggio della funzione obiettivo del problema specificato. Per semplicità di lettura si è deciso di effettuare una chiamata per ogni singola variabile.

Tale metodo richiede di specificare le seguenti informazioni (tali informazioni sono reperibili nel modello proposto in Sezione \ref{pt1:problem}):

\begin{itemize}
\item l'ambiente CPlex;
\item il problema sul quale esso dovrà operare;
\item il numero di nuove variabili da aggiungere;
\item il coefficiente della variabile in funzione obiettivo (vedi \ref{pt1:solver:datas});
\item il \english{lower bound};
\item l'\english{upper bound};
\item il tipo della variabile (intera, booleana, reale);
\item un'etichetta da associare alla variabile.
\end{itemize}

Al termine della computazione del metodo \texttt{setObjectiveFunction} le variabili all'interno del problema CPlex sono cosi memorizzate:

\begin{tabular}{|c|c|c|c|c|c|c|c|c|c|c|c|c|c}
\hline 
$y_{0,1}$ & $y_{0,2}$ & $\cdots$ & $y_{0,\left|N\right|}$ & $y_{1,0}$ & $y_{1,2}$ & $ \cdots $ & $y_{2,\left|N\right|}$ & $\cdots$ & $y_{\left|N\right|,1}$ & $y_{\left|N\right|,2}$ & $\cdots$ & $y_{\left|N\right|,\left|N\right| - 1}$ & $\cdots$\\ 
\hline 
\end{tabular}

\begin{tabular}{c|c|c|c|c|c|c|c|c|c|c|c|c|c|}
\hline 
$\cdots$ & $x_{0,1}$ & $x_{0,2}$ & $\cdots$ & $x_{0,\left|N\right|}$ & $x_{1,0}$ & $x_{1,2}$ & $ \cdots $ & $x_{2,\left|N\right|}$ & $\cdots$ & $y_{\left|N\right|,1}$ & $x_{\left|N\right|,2}$ & $\cdots$ & $x_{\left|N\right|,\left|N\right| - 1}$\\ 
\hline 
\end{tabular}  

\subsection[Caricamento vincoli]{caricamento vincoli}
\label{pt1:solver:constraint}
I vincoli sono aggiunti al modello CPlex tramite la chiamata al metodo \texttt{CPXaddrows} il quale richiede di specificare i seguenti parametri:

\begin{itemize}
\item gli indici delle variabili coinvolte all'interno del vettore delle variabili di CPlex;
\item i coefficienti delle variabili;
\item il senso del vincolo ($\le$, $=$, $\ge$);
\item il termine noto.
\end{itemize}

Le ultime tra informazioni sono banalmente ricavabili all'interno del modello esposto in Sezione \ref{pt1:problem} mentre i metodi \texttt{setConstaint\_*} all'interno della classe \texttt{TSPSolver} illustrano il processo di selezione delle variabili all'interno dell'ambiente di CPlex.

\subsection[Esecuzione]{esecuzione}
\label{pt1:solver:esecution}
Il metodo \texttt{resolver} della classe \texttt{TSPSolver} ha i seguenti compiti:

\begin{itemize}
\item coordinare il processo di caricamento dati;
\item coordinare il processo di impostazione dell'ambiente CPlex (funzione obiettivo, vincoli, tipologia di problema [max, min]);
\item risolvere il problema;
\item produrre il \english{file} con il problema (.lp) ed il \english{soluzione};
\item stampare a video il tempo (ms) di computazione speso per la risoluzione del problema.
\end{itemize}