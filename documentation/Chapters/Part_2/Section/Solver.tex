%-------------------------------------------------
%
% Solver.tex
%
%------------------------------------------------
\section[Il risolutore]{il risolutore}
\label{pt2:solver}
Il programma con l'implementazione della metaeuristica è composta dalle seguenti classi.
\newpage

\dirtree{%
.1 metaheuristic.
.2 src.
.3 header.
.4 object.
.5 Chromosome.h.
.5 Population.h.
.4 solver.
.5 GeneticAlgorithm.h.
.3 source.
.4 object.
.5 Chromosome.cpp.
.5 Population.cpp.
.4 solver.
.5 GeneticAlgorithm.cpp.
.2 main.cpp.
.2 makefile.
}

Le classi \texttt{Chromosome} e \texttt{Population} sono classi di servizio all'algoritmo e rappresentato rispettivamente un singolo cromosoma ed una popolazione.

La classe più importante è \texttt{GeneticAlgorithm} che racchiude in sé l'algoritmo illustrato in Sezione \ref{pt2:design}, ed è composta dai seguenti metodi:

\dirtree{%
.1 GeneticAlgorithm.
.2 private.
.3 readProblemDatas.
.3 getCost.
.3 getBetterChromosome.
.3 contains.
.3 fitnessFunction.
.3 initialize.
.3 crossover.
.2 public.
.3 resolve.
.3 getObjectiveValue.
}

I metodi \texttt{readProblemDatas} e \texttt{getCost} sono le medesime implementazioni che ritroviamo nel risolutore CPlex e servono a virtualizzare la matrice associata al grafo.

Il metodo \texttt{getBetterChromosome} restituisce il cromosoma con il miglior valore di \english{fitness} data una popolazione e viene utilizzato per estrarre i due \english{parent}.

Il metodo \texttt{fitnessFunction} calcola il valore di \english{fitness} dato un cromosoma.

Mentre il metodo \texttt{initialize} serve per inizializzare la popolazione iniziale prima di eseguire le evoluzioni genetiche.

Il metodo \texttt{crossover} esegue l'operazione di \english{crossover} mentre il metodo \texttt{resolve} è colui che coordina l'avanzamento delle evoluzioni.