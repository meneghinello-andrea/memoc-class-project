%-------------------------------------------------
%
% Design.tex
%
%------------------------------------------------
\section[Progettazione dell'algoritmo]{progettazione dell'algoritmo}
\label{pt2:design}
La progettazione dell'algoritmo è iniziata progettando la rappresentazione delle soluzioni del problema all'interno dell'algoritmo, ossia i \keyword{cromosomi} dell'algoritmo genetico.

Per il problema in questione si è scelta la \english{path representation} ossia ogni cromosoma è composto da una sequenza di \keyword{geni} dove ogni gene rappresenta un singolo foro da effettuare sulla piastra. Leggendo in sequenza i geni di ogni cromosoma otteniamo un possibile cammino hamiltoniano sul grafo in questione.

\begin{table}
\centering
\begin{tabular}{|c|c|c|c|c|c|c|c|c|c|c|c|c|c|}
\hline 
0 & 1 & 2 & 3 & 4 & 5 & 6 & 7 & 8 & 9 & 10 & 11 & 12 & 0\\
\hline 
\end{tabular}
\caption{Esempio di cromosoma in un grafo a 13 punti}
\end{table}

Il cromosoma presenta il gene 0 sia in principio che in fine del cromosoma in quanto si vuole che la testina ritorni nella posizione iniziale prima di iniziare ad operare su di una nuova piastra.

Dal momento che ogni permutazione dei geni di un cromosoma (ad eccezione del gene in posizione 0 e posizione $n + 1$) rappresenta una possibile soluzione del problema di partenza, la popolazione iniziale si è scelta mescolando casualmente il ``cromosoma zero'' dalla posizione 1 alla posizione $n + 1$ esclusa (garantendo che primo del ultimo gene rimangano inalterati).

Come \keyword{funzione di \english{fitness}} si è implementato l'inverso della funzione obiettivo illustrata in Sezione \ref{pt1:problem}; in questo modo minore sarà il valore della funzione obiettivo maggiore sarà il valore del valore di \english{fitness} associato ad un cromosoma, garantendogli cosi maggior probabilità di sopravvivenza.

Per la scelta dei due cromosomi \english{parent}, che dovranno generare un nuovo cromosoma per la generazione successiva (\english{offspring}), all'interno dell'i-esima evoluzione della popolazione si è ricorsi al metodo \keyword{torneo-n} cosi da evitare di privilegiare eccessivamente gli individui migliori di ogni generazione. Questo dovrebbe impedire di condurre la popolazione verso un massimo locale.

Per l'operazione di generazione di un nuovo figlio, operazione di \english{\keyword{crossover}} si è adottato l'operatore di \english{\keyword{order crossover}}. Ossia si selezionano due posizioni casuali all'interno del cromosoma padre e si ricopiano i geni in tali posizioni nelle medesime posizioni del figlio, mentre le rimanenti posizioni si riempiono con i geni assenti ma nello stesso ordine con cui compaiono all'iterno della madre.

Nella Tabella 2.2 viene fornito un esempio di \english{order crossover} con indici (4,9).

L'algoritmo è lasciato avanzare per un numero di evoluzioni prederminato dall'utente che lo avvia.

\begin{table}
\centering
\begin{tabular}{|c|c|c|c|c|c|c|c|c|c|c|c|c|c|}
\hline 
0 &  1 &  2 &  3 & 4 & 5 & 6 & 7 & 8 & 9 & 10 & 11 & 12 & 0\\
\hline
0 & 12 & 11 & 10 & 9 & 8 & 7 & 6 & 5 & 4 &  3 &  2 &  1 & 0\\
\hline
0 & 12 & 11 & 10 & 4 & 5 & 6 & 7 & 8 & 9 &  3 &  2 &  1 & 0\\
\hline
\end{tabular}
\caption{Esempio di order crossover (padre - madre - figlio)}
\end{table}

Con probabilità del 40\% il figlio generato subirà una mutazione effettuata tramite inversione. Questa avviene selezionando due posizioni casuali nel figlio, comprese tra 1 ed $n$, e si invertono i geni in quelle posizioni.