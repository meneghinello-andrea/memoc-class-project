%------------------------------------------------
%
% Preamble.tex
%
% This file contains all the settings for a correct
% compilation of the document. Include this file
% before starting the document section.
%------------------------------------------------
\documentclass[paper=A4, fontsize=11pt, parskip=full]{scrreprt}

\usepackage{acronym}
\usepackage{amsmath}
\usepackage{array}
\usepackage{caption}
\usepackage{color}
\usepackage{dirtree}
\usepackage{eurosym}
\usepackage{fourier}
\usepackage{graphicx}
\usepackage{ifthen}
\usepackage{lastpage}
\usepackage{listings}
\usepackage{longtable}
\usepackage{multirow}
\usepackage{pdfpages}
\usepackage{sectsty}
\usepackage{siunitx}
\usepackage{subcaption}
\usepackage[utf8]{inputenc}
\usepackage[babel]{csquotes}
\usepackage[T1]{fontenc}
\usepackage[hidelinks]{hyperref}
\usepackage[english, italian]{babel}
\usepackage[nonumberlist, toc]{glossaries}
\usepackage[protrusion=true, expansion=true]{microtype}
\usepackage[automark, headsepline, footsepline, markuppercase]{scrpage2}

\allsectionsfont{\centering \normalfont\scshape}

\pagestyle{scrheadings}

\lehead{Memoc}
\rohead{TSP Solver}
\chead{}
\cfoot{\thepage{} di \pageref{LastPage}}

\setglossarystyle{altlist}
\makeglossaries

\newcommand{\english}[1]{\foreignlanguage{english}{\textit{#1}}}
\newcommand{\keyword}[1]{\textbf{#1}}
\newcommand{\printListOfContents}[1]{\ifthenelse{\equal{#1}{TRUE}}{\tableofcontents{}}{}}
\newcommand{\printListOfFigures}[1]{\ifthenelse{\equal{#1}{TRUE}}{\listoffigures{}}{}}
\newcommand{\printListOfTables}[1]{\ifthenelse{\equal{#1}{TRUE}}{\listoftables{}}{}}
\newcommand{\printGlossary}[1]{\ifthenelse{\equal{#1}{TRUE}}{\printglossary[title=Glossario]{}}{}}
\newcommand{\glossarySingolarTerm}[1]{\underline{\gls{#1}}}
\newcommand{\glossaryPluralTerm}[1]{\underline{\glspl{#1}}}
\newcommand{\attribute}[1]{\textsc{#1}}

\definecolor{mygreen}{rgb}{0,0.6,0}
\definecolor{mygray}{rgb}{0.5,0.5,0.5}
\definecolor{mymauve}{rgb}{0.58,0,0.82}

\lstset{ %
  backgroundcolor=\color{white},   % choose the background color; you must add \usepackage{color} or \usepackage{xcolor}
  basicstyle=\footnotesize,        % the size of the fonts that are used for the code
  breakatwhitespace=false,         % sets if automatic breaks should only happen at whitespace
  breaklines=true,                 % sets automatic line breaking
  captionpos=b,                    % sets the caption-position to bottom
  commentstyle=\color{mygreen},    % comment style
  extendedchars=true,              % lets you use non-ASCII characters; for 8-bits encodings only, does not work with UTF-8
  frame=single,                    % adds a frame around the code
  keepspaces=true,                 % keeps spaces in text, useful for keeping indentation of code (possibly needs columns=flexible)
  keywordstyle=\color{blue},       % keyword style
  language=C++,                    % the language of the code
  numbers=left,                    % where to put the line-numbers; possible values are (none, left, right)
  numbersep=5pt,                   % how far the line-numbers are from the code
  numberstyle=\tiny\color{mygray}, % the style that is used for the line-numbers
  rulecolor=\color{black},         % if not set, the frame-color may be changed on line-breaks within not-black text (e.g. comments (green here))
  showspaces=false,                % show spaces everywhere adding particular underscores; it overrides 'showstringspaces'
  showstringspaces=false,          % underline spaces within strings only
  showtabs=false,                  % show tabs within strings adding particular underscores
  stepnumber=2,                    % the step between two line-numbers. If it's 1, each line will be numbered
  stringstyle=\color{mymauve},     % string literal style
  tabsize=2,                       % sets default tabsize to 2 spaces
  title=\lstname                   % show the filename of files included with \lstinputlisting; also try caption instead of title
}